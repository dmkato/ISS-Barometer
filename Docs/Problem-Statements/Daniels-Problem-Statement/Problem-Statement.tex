\documentclass[onecolumn, draftclsnofoot,10pt, compsoc]{IEEEtran}
\usepackage{graphicx}
\usepackage{url}
\usepackage{setspace}
\usepackage{indentfirst}

\usepackage{geometry}
\geometry{textheight=9.5in, textwidth=7in}

\def \CapstoneTeamName{Manovacometer}
\def \CapstoneTeamNumber{ 20}
\def \GroupMemberOne{Daniel Kato}
\def \GroupMemberTwo{Nathan Shepherd}
\def \GroupMemberThree{Cade Raichart}
\def \CapstoneProjectName{ISS Barometer}
\def \CapstoneSponsorCompany{NASA}
\def \CapstoneSponsorPerson{Don Pettit}

\def \DocType{Problem Statement
				%Requirements Document
				%Technology Review
				%Design Document
				%Progress Report
}

\newcommand{\NameSigPair}[1]{\par
\makebox[2.75in][r]{#1} \hfil 	\makebox[3.25in]{\makebox[2.25in]{\hrulefill} \hfill		\makebox[.75in]{\hrulefill}}
\par\vspace{-12pt} \textit{\tiny\noindent
\makebox[2.75in]{} \hfil		\makebox[3.25in]{\makebox[2.25in][r]{Signature} \hfill	\makebox[.75in][r]{Date}}}}
%\renewcommand{\NameSigPair}[1]{#1}

%%%%%%%%%%%%%%%%%%%%%%%%%%%%%%%%%%%%%%%
\begin{document}
\begin{titlepage}
    \pagenumbering{gobble}
    \begin{singlespace}
    	\includegraphics[height=4cm]{coe_v_spot1}
        \hfill
        %\includegraphics[height=4cm]{CompanyLogo}
        \par\vspace{20pt}
        \centering
        \scshape{
            \huge CS Capstone \DocType \par
            {\large\today}\par
            \vfill
            \textbf{\Huge\CapstoneProjectName}\par
            \vfill
            {\large Prepared for}\par
            \Huge \CapstoneSponsorCompany\par
            \vspace{10pt}
            {\Large\NameSigPair{\CapstoneSponsorPerson}\par}
            \vspace{10pt}
            {\large Prepared by }\par
            Group\CapstoneTeamNumber\par
            \CapstoneTeamName\par
            \vspace{10pt}
            {\Large
                \NameSigPair{\GroupMemberOne}\par
                \NameSigPair{\GroupMemberTwo}\par
                \NameSigPair{\GroupMemberThree}\par
            }
            \vfill
        }
        \begin{abstract}
        	When the International Space Station (ISS) has a leak, they use barometers and lookup tables to determine the amount of time they have before they need to evacuate.
					Unfortunately, the ISS is down to 2 barometers, but have 8 - 10 iPad Air 2's on board.
					Our goal is to create an application that can aid the astronauts on the ISS in calculating the amount of time they have before they need to evacuate the ISS in the event of a leak.\par
					To accomplish this we will create an interface to clearly display the current pressure, the starting pressure, and the dP/dT along with a live graph of the change in pressure over time.
					We will also allow the user to change settings like the number of significant digits displayed, the amount and granularity of the data displayed in the graph, and orientation
        \end{abstract}
    \end{singlespace}
\end{titlepage}
\newpage
\pagenumbering{arabic}
\tableofcontents
% 7. uncomment this (if applicable). Consider adding a page break.
%\listoffigures
%\listoftables
\clearpage

\section{Definition of the Problem}
	Currently, when the International Space Station (ISS) experiences rapid depressurization due to either a breach of the hull or a malfunctioning valve, they use old 1960s era Russian barometers called \textit{manovacometers} and lookup tables to determine the amount of time they have before the pressure drops below the safe pressure limit of 520mmHg and the crew must abandon the ISS.
	Unfortunately, because the barometers in use are no longer being manufactured, they are unable to get replacements when one breaks.
	As a result, the ISS is down to 2 barometers on board and have a need for more to aid in the event of a leak.\par
	In the event of rapid depressurization, the astronauts are interested in the change in pressure per second (dP/dT) as this is the number they lookup to find out how much time they have left.
	They are also interested in the initial pressure upon starting the application, the current pressure, and the current change in pressure.

\section{Description of the Problem}
  Rapid depressurization occurs on the International Space Station when debris damages the hull resulting in a breach, or when a valve is malfunctioning and allowing pressure to escape the cabin.

\section{Proposed Solution}
  To solve the problem described above, an iOS application will be developed to give the astronauts aboard the ISS the tools needed to find out how much time they have before they need to evacuate the space station in the even of a rapid depressurization event.
  This application will be written in the Swift language and will be targeted to mainly the iPad Air 2, but will be compatible with iPhones running iOS 10 - 11.\par

  The application will have 2 pages: a main page displaying the desired statistics, and a settings page to allow the user to change the way the data is displayed.
  Both pages will be compatible with both portrait and landscape view modes, but not via the conventional method of tilting to change orientation as there is no gravity in space.
  To overcome this we will add a button that when clicked will rotate the display.

  \subsection{Main Page}
    The main page will have to main sections: a top section containing 3 values of interest to the astronauts and a bottom section containing a live graph of dP/dT.
    The top section will contain the current pressure upon in mmHg, the current time, and seconds/mmHg of pressure loss accurate to 2 significant digits.
    The current pressure will be set whenever a user pressed the 'Measure Pressure' button and a timestamp will be recorded with each press of the button.
    Pressing the 'Measure Pressure' button will not overwrite the previously measured pressure, but add it to a log that will display all measured values.
    Each of these numbers will be displayed with units in clear legible text near the top of the page.
    It was noted that black text on white background is preferred.paragraph\par

    The bottom section will contain a running graph of dP/dT.
    This graph will display pressure as a function of time, and capture data at least every 2 seconds.
    Without being touched, the graph will grow as data points are added and the graph will end up being clipped on the left side as data is added to the right.
    The user should be able to pinch to zoom into parts of the graph, and when zoomed in, the graph will not move so as to not interrupt the viewing of data.
    The slope should be recorded and displayed next to the graph.

  \subsection{Settings Page}
  The settings page will be accessible by a standard gear style settings button in the bottom left.
  The settings will include options for the significant digits for each number displayed, displaying the app in landscape or portrait mode, and displaying the graph as a running window showing only the most current measurements or displaying the entire graph.


\section{Performance Metrics}

\end{document}
